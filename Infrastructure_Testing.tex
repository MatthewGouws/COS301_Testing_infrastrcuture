\documentclass[12pt]{article}
\usepackage{graphicx} %Required for diagrams
\usepackage{bookmark}
\usepackage{hyperref}

\begin{document}

\begin{titlepage}
	\begin{center}
		
		\begin{figure}[t]
			\centering
			\includegraphics[width=350px]{UP_Logo.png}
		\end{figure}
		
		% Title
		\textsc{\LARGE COS301 Mini Project Testing \newline\newline Infrastructure}
		
		%\begin{minipage}{0.4\textwidth}
		\begin{flushright} \large
			Matthew Gouws \emph{u11008602} \newline
			Andrew Parkes \emph{u12189139} \newline
			 Axel Ind \emph{u12063178} \newline
			 Patience Mtsweni \emph{u11116774} \newline
			 Khathutshelo Shaun Matidza\emph{u11072157} \newline
			 \emph{uxxxxxxxx} \newline
			 \emph{uxxxxxxxx} \newline
			 \emph{uxxxxxxxx} \newline
		\end{flushright}
		%\end{minipage}
		
		\vfill
		
	Here's a link to \href{https://github.com/MatthewGouws/COS301_Testing_infrastructure}{Github}.\\
	\url{https://github.com/MatthewGouws/COS301_Testing_infrastructure}

	\vfill

	{\large Version 0.1-alpha}
	\\
	{\large \today}		
		
		
	\end{center}
\end{titlepage}


\section{History}
\begin{tabular}{|p{3cm}|p{5cm}|p{6cm}|}

\hline
Date & Version & Description\\ %NOTE: Necessary for [updated by] ?
\hline
21-04-2015 & Version 0.1 & Document Template Created\\ %April 21 - Matthew (Took the date that the document was on github)
\hline
22-04-2015 & Version 0.1.1 & Added Authorization for B\\ %April 22 - Patience (Took the date that the document was on github)
\hline
22-04-2015 & Version 0.1.2 & Added Authorization for A\\ %April 22- Shaun (Took the date that the document was on github)
\hline
22-04-2015 & Version 0.1.3 & Added Notification Table\\ %April 22- Matthew (Took the date that the document was on github)
\hline
22-04-2015 & Version 0.1.4 & Added introduction\\ %April 22- Andrew (Took the date that the document was on github)
\hline
22-04-2015 & Version 0.1.5 & Added uses cases for Buzz B\\ %April 22- Andrew (Took the date that the document was on github)
\hline
22-04-2015 & Version 0.1.6 & Added uses cases for Buzz 1\\ %April 22- Matthew (Took the date that the document was on github)
\hline
22-04-2015 & Version 0.1.7 & Fixed Authorization formatting\\ %April 22- Shaun (Took the date that the document was on github)
\hline

\end{tabular}

\newpage
\tableofcontents
\newpage

\section{Introduction} % added by Andrew but please read over it and make corrections
This document contains:
Part 1 the functional testing phase for each mid level parts Buzz A and Buzz B.
Each section with show the success or the failure of each part. This contains all violations of the contract requirements.
pre- and post- conditions should be tested for all the violation and the data structure requirements.
For all the testing, an analysis report of the percentage cases will be given that will depict the amount of work done and the successfulness of the sections in the implementation

Part 2 the non-functional testing phase.
This part contains the performance, scalability, maintainability, reliability, usability of the application and problems associate with the system.

\section{Purpose} % purpose seems to be in order
The purpose of this task was to test functionality provided by mid-level integration for infrastructure, which consisted of Notification, Authorization, Spaces and CSDS.


\section{Project Scope} % To be filled in at a later stage
The scope of the integration for infrastructure was to combine all functional teams code in a manner which could be used by top level integration. From what has been discovered and explained further in this document it shows that both teams A and B have failed to do so. Team A was very difficult to try and decipher. With missing dependencies, while Team B only had mock functionality.

\section{Functional} %specify the use cases as in the MasterSpecifications - Table format, Compare A vs B
\subsection{use cases and results}

\subsubsection{Authorization}%NOTE: This table consists of all the use-cases under authorization
\begin{tabular}{|p{4.5cm}|p{4.5cm}|p{4.5cm}|}

\hline
Use Case(s) & Buzz A & Buzz B \\ 
\hline
addAuthorizationRest - Adds an authorization restriction for a user role in a particular buzz space. & Only mock functionality,but can not run & Only Mock functionality, does not run\\ %
\hline
updateAuthorizationRest - Facilitates editing of authorization restrictions. & Only mock functionality,but does not run & Only Mock functionality, does not run\\ %
\hline
removeAuthorizationRest - Removes an authorization restriction for a user role from a buzz space. & Only mock functionality,but can not run & Only Mock functionality, does not run\\ %
\hline
getAuthorizationRest - Retrieves the authorization restriction to enable users to select a restriction to update. & Only mock functionality,but can not run & Only Mock functionality, does not run\\ %
\hline
isAuthorized - Queries the services a user may access in order to customize the UI for the user. & Only mock functionality,but can not run & Only Mock functionality, does not run\\ %
\hline

\end{tabular}
\subsubsection{Notification}%NOTE: This table consists of all the use-cases under authorization
\begin{tabular}{|p{4.5cm}|p{4.5cm}|p{4.5cm}|}

\hline
Use Case(s) & Buzz A & Buzz B \\ 
\hline
Daily Email - Sends Daily Email. & Could not npm install, Missing dependencies, thus would not run & Only Mock functionality, does not run \\ %
\hline
Delete Notification - Checks if the user should receive a notification & Could not npm install, Missing dependencies, thus would not run & Only Mock functionality, does not run\\ %
\hline
Edit Notification Settings - Edits the notifications  & Could not npm install, Missing dependencies, thus would not run & Only Mock functionality, does not run\\ %
\hline
Web Notification - returns a list of notifications for the specified user & Could not npm install, Missing dependencies, thus would not run & Only Mock functionality, does not run \\ %
\hline
Register For Notification - Allows a user to register for notifications on a thread, to specified users  & Could not npm install, Missing dependencies, thus would not run & Only Mock functionality, does not run \\ %
\hline
Standard Notification - When a user adds a new thread it sends notifications to a list of registered users & Could not npm install, Missing dependencies, thus would not run &  Only Mock functionality, does not run\\ %
\hline


\end{tabular}
\section{Non-functional A} %Please correct me if I am wrong on these headings % seems all in order
\subsection{Performance}
Stub - This will be added in the future
\subsection{Scalability}
Stub - This will be added in the future
\subsection{Maintainability}
Stub - This will be added in the future
\subsection{Reliability}
Stub - This will be added in the future
\subsection{Usability}
Stub - This will be added in the future
\subsection{Availability}
Stub - This will be added in the future
\subsection{Manageability}
Stub - This will be added in the future
\subsection{Security}
Stub - This will be added in the future
\subsection{Monitorability and Auditability}
Stub - This will be added in the future
\subsection{Integrability}
Stub - This will be added in the future

\section{Non-functional B} %Please correct me if I am wrong on these headings % seems all in order
\subsection{Performance}
The use of Nodejs allows the B Infrastructure system to run in a multithreaded manner with very high performance. Furthermore, if the final system were to make use of MongoDB (as specified), an appropriate performance increase would be achieved, indicating minimal-bottlenecking in the system overall.

The Integration Team also provided code which was highly streamlined, with no recursive methods or unnecessary process iterations. Suitable response times can be expected from the system for any reasonable number of concurrent users.

\subsection{Scalability}
By its nature, Nodejs and its subcomponents provides a high level of intrinsic scalability, due in no small part, to the streamlined use of client-server architecture. 

The work of the mid-level integration team included the appropriate use of core Nodejs functionality to create Service Objects, facilitating easy implementation of a single or several thousand Spaces. Multithreading is implemented and no limitting bounds are set on the potential permissible Space IDs.

At present every function at the midlevel operates with O(logn) complexity (as the number of tasks increase, the odds of a given task already having been performed and, thus needed no additional complex work, increases significantly) 

\subsection{Maintainability}
Although the code appears simplistic and highly maintainable, minimal commenting was provided by the Integration Team. The current code is human-readable, but in the pursuit of long-term maintainability by multiple parties, the lack of authors’ notes could be seen as a potential stumbling block to future programmers.

That notwithstanding, the code provided by the Integration Team clearly makes strong distinctions between individual components and would, for example, allow modifications to the registerOnBuzzSpace and closeBuzzSpace processes independently and with suitable modularisation for easy further modification by a third party.

\subsection{Reliability}
The Integration Team of Buzz B have implemented a highly reliable way to integrate Spaces. An appropriate message is logged when an access attempt is made and several comprehensive exceptions are thrown by the system when registerOnBuzzSpace, closeBuzzSpace and other processes fail.

It may be advisable, however, to increase the number of exceptions that may be thrown to better inform the system as to the problem that has occurred. For example: rather than stating: "BuzzSpace 7A is closed or does not exist." It may be prudent to have 2 errors: “BuzzSpace 7A is closed” and “BuzzSpace 7A does not exist”.


\subsection{Usability}
Given, the admittedly sparse, template provided by the Integration B Team, it is clear that Buzz B’s modularised functionality would be highly useable at mid-level. All modules are given descriptive names, all variable names are human-readable and meaningful, and programming logic is clear throughout. 

The only remark on improvement has already been mentioned in Maintainability and would involve the addition of more comments to better guide a first-time user.

\subsection{Availability}
One would expect the availability of the Buzz B System to be almost 100%. The code itself is minimal and, given appropriate decoupling from the implementation of Modules at a lower level, one could expect and changes that need to be made at the mid-level to be performed very quickly, while low-level changes to implementation could be accounted for suitably quickly.

Because Services are implemented as discrete objects, it would also significantly reduce downtime that may be caused by the failure of a single space, if that space could be recreated or reconnected to in a suitably dynamic and time-sensitive way.

\subsection{Manageability}
Given the fairly abstract and well-decoupled nature of Buzz B Infrastructure, the Manageability of the component is fairly high and would facilitate easy management of (unlikely) logical errors in the Buzz B system. 

Again, the one negative aspect of the emergent manageability of the system is the lack of commenting which makes the code unnecessarily hard to read and would limit the ability of the a programmer, either inspecting code manually or through a comment documentation system (e.g. Doxygen), to quickly grasp nuances in logic that may be affected by minor changes.

\subsection{Security}
Given that the log in functionality has not been properly implemented for Buzz be there is little-to-no security in place to protect the system.
\subsection{Monitorability and Auditability}
The use of console logging ensures and action-listeners mean that all activities can be monitored as and when they occur.
\subsection{Integrability}
The high modularisation implemented by Buzz B means that future integrability can be easily achieved.


\section{References}
Stub - This will be added in the future



\end{document}

