\documentclass[12pt]{article}
\usepackage{graphicx} %Required for diagrams
\usepackage{bookmark}
\usepackage{hyperref}

\begin{document}

\begin{titlepage}
	\begin{center}
		
		\begin{figure}[t]
			\centering
			\includegraphics[width=350px]{UP_Logo.png}
		\end{figure}
		
		% Title
		\textsc{\LARGE COS301 Mini Project Testing \newline\newline Infrastructure}
		
		%\begin{minipage}{0.4\textwidth}
		\begin{flushright} \large
			Matthew Gouws \emph{u11008602} \newline
			Andrew Parkes \emph{u12189139} \newline
			 Axel Ind \emph{u12063178} \newline
			 Patience Mtsweni \emph{u11116774} \newline
			 Khathutshelo Shaun Matidza\emph{u11072157} \newline
			 \emph{uxxxxxxxx} \newline
			 \emph{uxxxxxxxx} \newline
			 \emph{uxxxxxxxx} \newline
		\end{flushright}
		%\end{minipage}
		
		\vfill
		
	Here's a link to \href{https://github.com/MatthewGouws/COS301_Testing_infrastructure}{Github}.\\
	\url{https://github.com/MatthewGouws/COS301_Testing_infrastructure}

	\vfill

	{\large Version 0.1-alpha}
	\\
	{\large \today}		
		
		
	\end{center}
\end{titlepage}


\section{History}
\begin{tabular}{|p{3cm}|p{5cm}|p{6cm}|}

\hline
Date & Version & Description\\ %NOTE: Necessary for [updated by] ?
\hline
21-04-2015 & Version 0.1 & Document Template Created\\ %April 21 - Matthew (Took the date that the document was on github)
\hline
22-04-2015 & Version 0.1.1 & Added Authorization for B\\ %April 22 - Patience (Took the date that the document was on github)
\hline
22-04-2015 & Version 0.1.2 & Added Authorization for A\\ %April 22- Shaun (Took the date that the document was on github)
\hline
22-04-2015 & Version 0.1.3 & Added Notification Table\\ %April 22- Matthew (Took the date that the document was on github)
\hline
22-04-2015 & Version 0.1.4 & Added introduction\\ %April 22- Andrew (Took the date that the document was on github)
\hline
22-04-2015 & Version 0.1.5 & Added uses cases for Buzz B\\ %April 22- Andrew (Took the date that the document was on github)
\hline
22-04-2015 & Version 0.1.6 & Added uses cases for Buzz 1\\ %April 22- Matthew (Took the date that the document was on github)
\hline
22-04-2015 & Version 0.1.7 & Fixed Authorization formatting\\ %April 22- Shaun (Took the date that the document was on github)
\hline
23-04-2015 & Version 0.1.8 & Added Space use cases\\ %April 23- Matthew(The original one) (Took the date that the document was on github)
\hline
23-04-2015 & Version 0.1.9 & Added CSDS use cases\\ %April 23- Ephiphania (Took the date that the document was on github)
\hline
24-04-2015 & Version 0.2.0 & Added availability and security non-functional requirements\\ %April 24- Shaun (Took the date that the document was on github)
\hline
24-04-2015 & Version 0.2.1 & Added Manageability and Reliability\\ %April 24- Andrew (Took the date that the document was on github)
\hline
24-04-2015 & Version 0.2.2 & Added Monitor ability and Auditability and integrability\\ %April 24- Andrew (Took the date that the document was on github)
\hline
24-04-2015 & Version 0.3 alpha & Proposed Alpha testing documentation\\ %April 24- Matthew (Took the date that the document was on github)
\hline
\end{tabular}

\newpage
\tableofcontents
\newpage

\section{Introduction} % added by Andrew but please read over it and make corrections
					   % This section read over by Matthew(The original one)
					   %fixed mid-level
This document contains:
Part 1 the functional testing phase for each mid-level parts Buzz A and Buzz B.
Each section will show the success or the failure of each part. This contains all violations of the contract requirements.
Pre- and post- conditions should be tested for all the violations and the data structure requirements.
For all the testing, an analysis report of the percentage cases will be given that will depict the amount of work done and the successfulness of the sections in the implementation.

Part 2 the non-functional testing phase.
This part contains the performance, scalability, maintainability, reliability, usability of the application and problems associated with the system.

\section{Purpose} % purpose seems to be in order
The purpose of this task was to test functionality provided by mid-level integration for infrastructure, which consisted of Notification, Authorization, Spaces and CSDS.


\section{Project Scope} % To be filled in at a later stage
The scope of the integration for infrastructure was to combine all functional teams' code in a manner which could be used by top level integration. From what has been discovered and explained further in this document it shows that both teams A and B have failed to do so. Team A was very difficult to try and decipher. With missing dependencies, while Team B only had mock functionality.

\section{Functional} %specify the use cases as in the MasterSpecifications - Table format, Compare A vs B
\subsection{use cases and results}

\subsubsection{Authorization}%NOTE: This table consists of all the use-cases under authorization
\begin{tabular}{|p{4.5cm}|p{4.5cm}|p{4.5cm}|}

\hline
Use Case(s) & Buzz A & Buzz B \\ 
\hline
addAuthorizationRest - Adds an authorization restriction for a user role in a particular buzz space. & Could not NPM start as stated in README due to missing dependencies, as a result could not run & Only Mock functionality, does not run\\ %
\hline
updateAuthorizationRest - Facilitates editing of authorization restrictions. &  Could not NPM start as stated in README due to missing dependencies, as a result could not run & Only Mock functionality, does not run\\ %
\hline
removeAuthorizationRest - Removes an authorization restriction for a user role from a buzz space. &  Could not NPM start as stated in README due to missing dependencies, as a result could not run & Only Mock functionality, does not run\\ %
\hline
getAuthorizationRest - Retrieves the authorization restriction to enable users to select a restriction to update. & Could not NPM start as stated in README due to missing dependencies, as a result could not run & Only Mock functionality, does not run\\ %
\hline
isAuthorized - Queries the services a user may access in order to customize the UI for the user. & Could not NPM start as stated in README due to missing dependencies, as a result could not run& Only Mock functionality, does not run\\ %
\hline

\end{tabular}
\subsubsection{Notification}%NOTE: This table consists of all the use-cases under authorization
\begin{tabular}{|p{4.5cm}|p{4.5cm}|p{4.5cm}|}

\hline
Use Case(s) & Buzz A & Buzz B \\ 
\hline
Daily Email - Sends Daily Email. & Could not npm install, Missing dependencies, thus would not run & Only Mock functionality, does not run \\ %
\hline
Delete Notification - Checks if the user should receive a notification & Could not npm install, Missing dependencies, thus would not run & Only Mock functionality, does not run\\ %
\hline
Edit Notification Settings - Edits the notifications  & Could not npm install, Missing dependencies, thus would not run & Only Mock functionality, does not run\\ %
\hline
Web Notification - returns a list of notifications for the specified user & Could not npm install, Missing dependencies, thus would not run & Only Mock functionality, does not run \\ %
\hline
Register For Notification - Allows a user to register for notifications on a thread, to specified users  & Could not npm install, Missing dependencies, thus would not run & Only Mock functionality, does not run \\ %
\hline
Standard Notification - When a user adds a new thread it sends notifications to a list of registered users & Could not npm install, Missing dependencies, thus would not run & Only Mock functionality, does not run\\ %
\hline


\end{tabular}

\subsubsection{Spaces}%NOTE: This table consists of all the use-cases under spaces
\begin{tabular}{|p{4.5cm}|p{4.5cm}|p{4.5cm}|}

\hline
Use Case(s) & Buzz A & Buzz B \\ 
\hline
Create Buzz Space - Creates and adds the buzz space to the activated list of buzz spaces. & Could not npm install, Missing dependencies, thus would not run & Only Mock functionality, does not run \\ 
\hline
Close Buzz Space - Receives buzz to close and then removes the buzz space from the list of activated buzz spaces. & Could not npm install, Missing dependencies, thus would not run & Only Mock functionality, does not run\\ %
\hline
Assign Administrator - Gets the user to be assigned to be administrator then checks if it is administrator and adds the user to the list of administrators.  & Could not npm install, Missing dependencies, thus would not run & Only Mock functionality, does not run\\ %
\hline
Remove Administrator - Receives the user to be removed then removes the user from the list of admin. & Could not npm install, Missing dependencies, thus would not run & Only Mock functionality, does not run\\ %
\hline
Is Administrator - Receives the user to be checked then searches the admin list for the user.  & Could not npm install, Missing dependencies, thus would not run & Only Mock functionality, does not run\\ %
\hline
Get User Profile - Searches for the user that is queried and returns the user searched for. & Could not npm install, Missing dependencies, thus would not run & Only Mock functionality, does not run\\ %
\hline
Register On Buzz Space - Registers the user on buzz spaces and stores the user in the database. & Could not npm install, Missing dependencies, thus would not run & Only Mock functionality, does not run\\ %
\hline


\end{tabular}

\subsubsection{The Buzz-Data-Sources Module}%NOTE: This table consists of all the use-cases under CSDS
\begin{tabular}{|p{4.5cm}|p{4.5cm}|p{4.5cm}|}

\hline
Use Case(s) & Buzz A & Buzz B \\ 
\hline
Login and administrative user - The system should authenticate against the CS DataSources within which authentication credentials are currently stored. & Gets connection to DataSource, Authenticate user against the DataSource and userID is returned. & Gets connection to DataSource, Authenticate user against the DataSource and userID is returned.\\ %
\hline
getUsersRolesForModule - This function is to query the user roles for a particular user. & The function successfully meets the requirements it has local copy to the userID, ModuleID and roles array. It returns user roles for module request. &  The function fails to meets the requirements it has local copy to the userID, ModuleID and roles array but it \emph{does not} return user roles for module request.\\ %
\hline
getUsersWithRole - It retrieves all the users that have a particular role like a teachingAssistant role for a particular module. & The function fails to meets the requirements has local copy to the userID, ModuleID and roles array but it \emph{does not} return user roles for module request. & The function fails to meets the requirements has local copy to the userID, ModuleID and roles array. It \emph{does not} returns user roles for module request\\ %
\hline

\end{tabular}


\section{Non-functional} %Please correct me if I am wrong on these headings % seems all in order
\subsection{Performance}
%Axel please add here
The use of Nodejs allows the B Infrastructure system to run in a multithreaded manner with very high performance. Furthermore, if the final system were to make use of MongoDB (as specified), an appropriate performance increase would be achieved, indicating minimal-bottlenecking in the system overall.

The Integration Team also provided code which was highly streamlined, with no recursive methods or unnecessary process iterations. Suitable response times can be expected from the system for any reasonable number of concurrent users.


As stated for the B system, A would run similarly in performance due to the use of NodeJS and MongoDB, With the code very difficult to track down and run, the true performance could not be tested for the system, but given a decent entry level server would be expected to handle multiple client connections at once.
\subsection{Scalability}
By its nature, Nodejs and its subcomponents provides a high level of intrinsic scalability, due in no small part, to the streamlined use of client-server architecture. 

The work of the mid-level integration teams for both A and B included the appropriate use of core Nodejs functionality to create Service Objects, facilitating easy implementation of a single or several thousand Spaces. Multithreading is implemented and no limitting bounds are set on the potential permissible Space IDs.

At present every function at the midlevel operates with O(logn) complexity (as the number of tasks increase, the odds of a given task already having been performed and, thus needed no additional complex work, increases significantly) 

\subsection{Maintainability}
Although the code appears simplistic and highly maintainable, minimal commenting was provided by the Integration Team. The current code is human-readable, but in the pursuit of long-term maintainability by multiple parties, the lack of authors’ notes could be seen as a potential stumbling block to future programmers.

That notwithstanding, the code provided by the Integration Team clearly makes strong distinctions between individual components and would, for example, allow modifications to the registerOnBuzzSpace and closeBuzzSpace processes independently and with suitable modularisation for easy further modification by a third party.

Due to the fact that the code is very difficult to access in the repository for A it is highly unlikely it can be called maintainable. However unit tests seem to have been completed and thus would be easy to add specific extra cases to the code while ensuring it does not break the known working code.
\subsection{Reliability}
Due to the lack in all aspects of Buzz B the reliability is non-existent due to missing code, mock functions and not being able to run the packages.

The Integration Team of Buzz B have implemented a highly reliable way to integrate Spaces. An appropriate message is logged when an access attempt is made and several comprehensive exceptions are thrown by the system when registerOnBuzzSpace, closeBuzzSpace and other processes fail.

It may be advisable, however, to increase the number of exceptions that may be thrown to better inform the system as to the problem that has occurred. For example: rather than stating: "BuzzSpace 7A is closed or does not exist." It may be prudent to have 2 errors: “BuzzSpace 7A is closed” and “BuzzSpace 7A does not exist”.

\subsection{Usability}
Given, the admittedly sparse, template provided by the Integration B Team, it is clear that Buzz B’s modularised functionality would be highly useable at mid-level. All modules are given descriptive names, all variable names are human-readable and meaningful, and programming logic is clear throughout. 

The only remark on improvement has already been mentioned in Maintainability and would involve the addition of more comments to better guide a first-time user.


Due to the fact that the code is barely traceable to where it is located, it is very difficult to use this package. Insight into the development would be required to allow a user to be able to use the code correctly and how to use the code before production could begin.
\subsection{Availability}
One would expect the availability of the Buzz B System to be almost 100 percent. The code itself is minimal and, given appropriate decoupling from the implementation of Modules at a lower level, one could expect and changes that need to be made at the mid-level to be performed very quickly, while low-level changes to implementation could be accounted for suitably quickly.

Because Services are implemented as discrete objects, it would also significantly reduce downtime that may be caused by the failure of a single space, if that space could be recreated or reconnected to in a suitably dynamic and time-sensitive way.

Considering the fact that the system was going to be hosted online on successful completion, its availability would've been archived since it would be in an operable and committable state at any random time. However this is not the case with the current state of the system.

\subsection{Manageability}
Given the fairly abstract and well-decoupled nature of Buzz B Infrastructure, the Manageability of the component is fairly high and would facilitate easy management of (unlikely) logical errors in the Buzz B system. 

Again, the one negative aspect of the emergent manageability of the system is the lack of commenting which makes the code unnecessarily hard to read and would limit the ability of the a programmer, either inspecting code manually or through a comment documentation system (e.g. Doxygen), to quickly grasp nuances in logic that may be affected by minor changes.

The manageability of part A is by no means possible. Without being able to get Buzz to run it is impossible to have any manageability in the system. For Buzz B it is highly manageable in its current state due to all the mock functions but since the code does not produce any significant functionality it is thus overlooked.
\subsection{Security}
%Shaun initially added this
Authorization is the module that was supposed to provide the degree of security in this system. Both authorization A and B  failed to run due to missing dependencies and only mock functionality provided for the other. Considering this, resistance to, or protection from, harm failed.
\subsection{Monitor ability and Auditability}
In both systems Buzz A and B there does not seem to be any logs or error messages to help with monitoring of the system. Since there is not expected outcome for each we are unable to test to see how auditable the systems are and thus both systems are not monitor able nor are they auditable.
\subsection{Integrability}
Since there is none of any of the above parts the integrability of the system is not possible. With unreliable systems that are not manageable, without it being usable the system in both Buzz spaces A and B they are not integrable

\section{References}
\begin{itemize}
\item \url{http://searchdatacenter.techtarget.com/definition/scalability }
\item \url{http://stackoverflow.com/questions/9420014/what-does-it-mean-scalability}
\item \url{http://en.wikipedia.org/wiki/Reliability,_availability_and_serviceability_%28computing%29}
\item \url{http://whatis.techtarget.com/definition/Reliability-Availability-and-Serviceability-RAS}
\end{itemize}




\end{document}

